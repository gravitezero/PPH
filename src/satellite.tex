%De l'importance et de la place de la communication externe dans l'associatif
%TODO remplacer le titre par celui de la fiche contrat pédagogique.

\section {Introduction}

    Depuis bientôt 4 ans, je suis mes études à l'Insa.
    La première chose que l'on m'as dite sur cette école, avant même son nom, ce fût l'importance de la composante humaine de la formation par rapport à d'autres établissements. Ce travail en est d'ailleurs la preuve.

    Depuis bientôt 4 ans, j'ai eu l'occasion de travailler dans plusieurs associations : le Ski Club, l'AEDI, le BDE, les 24h pour ne citer que les plus importantes.
    Toutes ces experiences ont en commun ma passion pour les arts graphiques et la communication visuelle.

        
    
\section{But de la communication exterieure}

Le but premier de la communication est de promouvoir les services, ou actions de l'association.
Dans le cas du ski club, ce sera les sorties proposé et la location, pour les 24h, ce sera la festival, et autres soirées, pour le bde, ce sera les services et évenement organisé.
Mais on peut prendre aussi l'exemple du forum Rhone Alpe, de la pluspart des association de département etc ...

Tout le travail effectué en communication exterieur, est principalement tournée vers la promotion des autres actions faites par l'association.
Un seul contre exemple me viens en tête, il s'agit des cartes de voeux envoyé par les 24h au partenaires.
Mais, en un sens, il peut également s'agir de promotion, puisque ça contribue à garder de bonne relation, et à pérénniser les partenariats commerciaux.


\section{Mes experiences dans l'associatif}

    \subsection{Le ski club}
        
        Ayant vécu mon enfance en haute-savoie, j'ai été initié au ski. Ne voulant pas perdre cette habitude, je me suis inscrit dès ma première année comme membre actf du Ski club.
        L'ancien webmaster quittait l'Insa l'année de mon arrivé, je me suis donc proposé de reprendre le site de l'association pour le maintenir.
        Ce fut ma première experience relative à la communication dans l'associatif.
        Et pour le ski club, cela marqua un changement dans la visibilité en ligne.
        
        Le site contient des informations concernant l'organisation des sorties, du weekend, ainsi que sur la location ou le fartage de materiel.
        
        
        Je ne fais plus parti du skiclub, le site à donc été remplacé par les membres actifs.
        
        %TODO corriger et compléter
        
    \subsection{L'AEDI}
        
        Il serait faux de dire que je suis membre de l'AEDI. Je n'ai que proposé d'aider en cas de besoin.
        J'ai par exemple déssiné un visuel de T-Shirt distribué aux "bizuth" lors du weekend d'intégration.
        J'ai également déssiné différentes affiches pour des évenements tels que le barbecue de fin d'année...
        
        %TODO corriger et compléter
        
    \subsection{Les 24 heures de l'Insa}
        
        C'est la deuxième année que je suis dans l'équipe communication des 24h de l'insa, et c'est probablement pour cette association que j'ai fourni le plus de travail.
        Les 2 annès j'ai dessiné beaucoup de visuels qui ont été réutilisé sur différents supports, tels que les programmes, les gobelets, les T-Shirts, les affiches etc ...
        J'ai également participé à l'élaboration des affiches journées et concerts.
        
        
    \subsection{Le BDE}
    
        J'ai également travaillé sur plusieurs projets avec l'équipe communication du bde.


\section{Résultats et conclusion tiré de ces expériences}

%TODO découper en sous parties

De manière général, dans l'associatif le travail fourni est important. Je pourrais estimé à 10\% le ratio de travaux qui seront effectivement publié.

Il y'a effectivement une part du travail qui est dédié à des essais, ou soumis à des selections ; et également une part du travail qui ne sera jamais aboutie, ou abandonné en cours de route.

La principale cause de ce gaspillage, est la méthode imployé : en effet, il est plus facile de travailler sur une production, de la finir, puis d'en commencer une autre, et ainsi de suite. Ce travail est parallélisé sur plusieurs personnes, pour plus de rapidité.

%TODO expliquer un peu plus
Je pense que le plus grand bénéfice que l'on pourrait tiré, c'est de s'obliger à établir une charte graphique claire, pour éviter de s'éparpiller en travaux inutiles.
Il serais plus judicieux, à mon sens, de travailler sur un ensemble de ressource graphique réutilisable plutôt que de travailler production par production.

Dans le cas du bde, cette charte graphique est présente, car la communication s'étale tout au long de l'année, et il est indispensable d'avoir une cohérence visuelle.
En revanche, dans le cas des 24h, du fait que chaque année, le thème change, la production graphique est entièrement à refaire.
C'est la raison pour laquelle, pendant les 2 ans où j'ai participé à cette équipe, le travail de chacun s'éparpillait dans un nombre incroyable de travaux très interressant, mais peu voir pas cohérent ensemble, donc inutilisable.


% ////////////////////// OLD

 mais la communication externe n'est pas toujours à la hauteur pour promouvoir les efforts fourni par les autres services.

Exemples :

    bde : comparativement à la médiatisation que l'on peut voir dans la pluspart des entrerpises, le bde a un service de communication sous-estimé et qui, je pense, devrait être plus important afin de mieux faire connaitre les services proposé aux étudiants.

    24h : La communication au 24h est plus importante, mais elle n'est souvent pas assez organisé pour avoir un plan de com sur plusieurs années.
    
Elle est souvent faites trop vite et tard pour être vraiment reflechi et occuper une place importante.
Cependant, contrairement au bde ou il faudrait informer le campus, les 24h benefissient déjà d'une réputation suffisante pour assurer un public suffisant, et la communication à plus une fonction informative qu'attractive.

% /////////////////////// \OLD



\section{Portefolio}

    Je présente ici différents travaux qui illustrent le mieux mon travail, et les évolutions que suis un projet.
    %TODO à remanier.

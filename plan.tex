De l'importance et de la place de la communication externe dans l'associatif


\section {Introduction}

    Depuis bientôt 4 ans, je suis mes études à l'Insa.
    La première chose que l'on m'as dite sur cette école, avant même son nom, ce fût l'importance de la composante humaine de la formation par rapport à d'autres établissements. Ce travail en est d'ailleurs la preuve.

    Depuis bientôt 4 ans, j'ai eu l'occasion de travailler dans plusieurs associations : le Ski Club, l'AEDI, le BDE, les 24h pour ne citer que les plus importantes.
    Toutes ces experiences ont en commun ma passion pour les arts graphiques et la communication visuelle.

\section{Mes experiences dans l'associatif}

    \subsection{Le ski club}
        
        Ayant vécu mon enfance en haute-savoie, j'ai été initié au ski. Ne voulant pas perdre cette habitude, je me suis inscrit dès ma première année comme membre actf du Ski club.
        L'ancien webmaster quittait l'Insa l'année de mon arrivé, je me suis donc proposé de reprendre le site de l'association pour le maintenir.
        Ce fut ma première experience relative à la communication dans l'associatif.
        Et pour le ski club, cela marqua un changement dans la visibilité en ligne.
        
        Le site contient des informations concernant l'organisation des sorties, du weekend, ainsi que sur la location ou le fartage de materiel.
        
        
        Je ne fais plus parti du skiclub, le site à donc été remplacé par les membres actifs.
        
        %TODO corriger et compléter
        
    \subsection{L'AEDI}
        
        Il serait faux de dire que je suis membre de l'AEDI. Je n'ai que proposé d'aider en cas de besoin.
        J'ai par exemple déssiné un visuel de T-Shirt distribué aux "bizuth" lors du weekend d'intégration.
        J'ai également déssiné différentes affiches pour des évenements tels que le barbecue de fin d'année...
        
        %TODO corriger et compléter
        
    \subsection{Les 24 heures de l'Insa}
        
        C'est la deuxième année que je suis dans l'équipe communication des 24h de l'insa, et c'est probablement pour cette association que j'ai fourni le plus de travail.
        Les 2 annès j'ai dessiné beaucoup de visuels qui ont été réutilisé sur différents supports, tels que les programmes, les gobelets, les T-Shirts, les affiches etc ...
        J'ai également participé à l'élaboration des affiches journées et concerts.
        
        
    \subsection{Le BDE}
    
        J'ai également travaillé sur plusieurs projets avec l'équipe communication du bde.
        
    
    
    %TODO STOP SHOWING OFF


\section{Ce que j'ai tiré de ces experiences} %TODO trouver un titre un peu plus "titre"

De manière général, dans l'associatif le travail fourni est important, mais la communication externe n'est pas toujours à la hauteur pour promouvoir les efforts fourni par les autres services.

Exemples :

    bde : comparativement à la médiatisation que l'on peut voir dans la pluspart des entrerpises, le bde a un service de communication sous-estimé et qui, je pense, devrait être plus important afin de mieux faire connaitre les services proposé aux étudiants.

    24h : La communication au 24h est plus importante, mais elle n'est souvent pas assez organisé pour avoir un plan de com sur plusieurs années.
Elle est souvent faites trop vite et tard pour être vraiment reflechi et occuper une place importante.
Cependant, contrairement au bde ou il faudrait informer le campus, les 24h benefissient déjà d'une réputation suffisante pour assurer un public suffisant, et la communication à plus une fonction informative qu'attractive.

\section{Portefolio}

    Je présente ici différents travaux qui illustrent le mieux mon travail, et les évolutions que suis un projet.
    %TODO à remanier.
